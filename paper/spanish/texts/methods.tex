\section{Métodos}

Boscá et. al. \cite{Bosca15} mencionan que existe una similitud entre FHIR y enfoques como el utilizado en openEHR. Dado que los recursos de FHIR y los arquetipos de openEHR definen patrones reutilizables para la descripción precisa de la información clínica. Sin embargo, los trabajos de colaboración entre las comunidades de FHIR y openEHR \cite{Collaboration} no consiguieron relacionar recursos y arquetipos existentes que representan conceptos clínicos similares debido a los principios de diseño diferentes utilizados por ambas comunidades. Siendo la principal diferencia que los arquetipos esperan representar la mayoría del contenido clínico, mientras que los recursos solo contienen la información clínica utilizada más común.

Sin embargo, como un recurso FHIR contiene un conjunto de elementos de datos estructurados descriptos en la definición de su tipo \cite{FHIRResource}, y un arquetipo openEHR constituye una encapsulación de un conjunto de puntos de datos pertencientes a un contenido de dominio, expresado en términos de restricciones del modelo de información de referencia openEHR \cite{openEHRArchetype}, una forma de encontrar correspondencia entre un recurso FHIR y un arquetipo openEHR es encontrar las relaciones entre los elementos de datos del recurso FHIR y los puntos de datos del arquetipo openEHR. El enfoque propuesto en este trabajo para encontrar estas relaciones es crear un nuevo arquetipo openEHR a partir de la definición de un recurso FHIR existente.

Debido a que la definición de un elemento en un recurso FHIR incluye tipo de dato \cite{FHIRElement}, es un prerrequisito encontrar las equivalencias entre los tipos de datos FHIR y los tipos de datos openEHR.

La creación de un arquetipo openEHR a partir de la definición de un recurso FHIR utilizando las equivalencias de tipos de datos puede ser sencilla. Dentro de un editor de arquetipos como los citados en \cite{openEHRModellingTools}, se puede crear de forma manual el arquetipo openEHR modelando la misma estructura del recurso FHIR utilizando el tipo de dato openEHR donde corresponda teniendo en cuenta las equivalencias de tipos de datos. Una alternativa diferente se presenta en este trabajo, consistente en la creación a través de un proceso automatizado. Las ventajas de esta alternativa frente al tipo de creación manual son un tiempo de creación y un riesgo de introducir errores por factor humano menores.

\begin{figure*}[t]
  \centering
  \includegraphics[scale=0.6]{./images/solution}
  \caption{Proceso automatizado}
  \label{fig:solution}
\end{figure*}

\subsection{Equivalencias entre tipos de datos}

La definición de un elemento en un recurso FHIR incluye tipo de dato \cite{FHIRElement}. Por lo tanto, un prerequisito es encontrar equivalencias entre tipos de datos FHIR y tipos de datos openEHR. Las equivalencias permitirán crear arquetipos openEHR a partir de definiciones de recursos FHIR.

Dada la función \( f(x) \) que retorna el dominio de valores que se le puede asignar a un dato \( x \), \( A \) el conjunto de atributos del tipo de dato de openEHR \( o \), la equivalencia entre el tipo de dato primitivo de FHIR \( p \)  y el tipo de dato de openEHR \( o \) existe si se cumplen las condiciones de:

\begin{enumerate}
  \item \( \exists a \in A \land f(a) \supseteq f(p) \);
  \item el uso de \( o \) es el mismo uso de \( p \).
\end{enumerate}

Las equivalencias se encuentran al realizar una revisión manual exhaustiva del conjunto de tipos de datos primitivos de FHIR \cite{FHIRDataTypes} y del Modelo de Información de Tipos de Datos de openEHR \cite{openEHRDataTypes}.

Como ejemplo, el tipo de FHIR boolean y el tipo de openEHR DV\_BOOLEAN reúnen ambas condiciones, por lo tanto, son equivalentes:
\begin{enumerate}
  \item los valores true y false del dominio de valores del tipo boolean pueden ser almacenados dentro del atributo value del tipo DV\_BOOLEAN;
  \item el uso de DV\_BOOLEAN, el cual especifica que el tipo se usa para elementos que son datos verdaderamente booleanos, es el mismo uso que tiene el tipo de dato boolean.
\end{enumerate}

Solo los tipos de datos primitivos de FHIR requieren una equivalencia con los tipos de datos openEHR.


\subsection{Automated process}

The automated process is based on the formalization proposed in \cite{Maldonado09}, which uses a type system over a tree structure for the definition of the archetype section. The type system models the structural constraints imposed by the archetypes over the reference model. Type system basis is the constraint multiplicity list (CML). CML is a definition language that specifies the valid sequence of attribute node or archetype object childs within the definition.

The automated process has three stages, as shown in figure \ref{fig:solution}: abstraction, substitution and definition. The stages repeat themselves for each openEHR integration archetype to be created from each FHIR resource

\begin{figure}[h]
  \centering
  \includegraphics[scale=0.5]{./images/solution}
  \caption{Automated process.}
  \label{fig:solution}
\end{figure}

\subsubsection{Etapa de abstracción}

El recurso FHIR se modela dentro de un sistema de tipos similar al desarrollado en \cite{Maldonado09}. Cada elemento del recurso FHIR se abstrae por el tipo que describe su estructura. La definición de un tipo es de la forma:

\begin{align*}
T_t:=p_t\{h_t\}
\end{align*}

\noindent
donde \(T_t\) es el nombre del tipo \(t\), \(p_t\) es el predicado que describe los valores soportados por el tipo \(t\) y \(h_t\) es un CML que especifica los elementos hijos que puede tener el tipo \(t\).

La definición del recurso FHIR \(R\) con elementos \(E_1\), \dots , \(E_2\) se abstrae con la definición de tipo como sigue:

\begin{align*}
T_R:=es\_R\{T_{E_1}^{(min_{E_1} \colon max_{E_1})} \dots T_{E_2}^{(min_{E_2} \colon max_{E_2})}\}
\end{align*}

\noindent
donde \(T_{E_1}\) es el tipo que define el elemento \(E_1\), \(min_{E_1}\) y \(max_{E_1}\) son los límites inferior y superior respectivamente de las veces que se permite que el elemento \(E_1\) aparezca en el recurso \(R\). La diferencia con el CML introducido en \cite{Maldonado09} es que en la definición de un tipo no se utilizan las restricciones de longitud por no agregar información adicional a la abstracción de un recurso FHIR.

Para modelar vinculaciones, se extiende el sistema de tipos presentado en \cite{Maldonado09}, agregando definiciones de vinculación. Una definición de vinculación es de la forma:

\begin{align*}
[T_E] := CV
\end{align*}

\noindent
donde \(T_E\) es la definición del tipo del elemento al cual se le vincula el conjunto de valores \(CV\).

Por ejemplo, al considerar el recurso SimplePatient (simplificación del recurso FHIR Patient \cite{FHIRPatient}) que modela un paciente que tiene un solo elemento gender del tipo code \cite{FHIRDataTypes} vinculado al conjunto de valores AdministrativeGender \cite{FHIRAdministrativeGender}, se modela por el conjunto de tipos:

\begin{align*}
&T_{SimplePatient}:= \\
&\qquad es\_SimplePatient\{T_{SimplePatient.gender}^{(0:1)}\} \\
&T_{SimplePatient.gender}:= \\
&\qquad es\_gender\{T_{SimplePatient.gender.code}^{(1:1)}\} \\
&T_{SimplePatient.gender.code}:= \\
&\qquad es\_code\{\epsilon\} \\
&[T_{SimplePatient.gender.code}] := \\
&\qquad URL \footnotemark \\
\end{align*}

\footnotetext{https://www.hl7.org/fhir/valueset-administrative-gender.html}
Un conjunto de tipos que define un recurso FHIR se denomina esquema FHIR en este trabajo.


\subsection{Etapa de sustitución}
A partir de los esquemas FHIR se generan los esquemas openEHR. Estos son conjuntos de tipos de openEHR definidos formalmente para modelar los recursos FHIR. Dicha generación se realiza al sustituir los tipos de datos de FHIR por los de openEHR siguiendo las equivalencias encontradas. Ambos esquemas conservan la jerarquía estructural y la semántica.

Continuando con el ejemplo del recurso SimplifiedPatient al realizar el reemplazo de los tipos de FHIR por sus equivalentes de openEHR, se obtiene el esquema openEHR ilustrado en la figura \ref{fig:substitution}.

\begin{figure}
  \includegraphics[scale=0.5]{./images/substitution}
  \caption{Esquema openEHR del recurso SimplifiedPatient}
  \label{fig:substitution}
\end{figure}


\subsubsection{Etapa de definición}

Se crea un arquetipo openEHR usando Archetype Definition Language (ADL) \cite{openEHRADL} según las definiciones de un esquema openEHR.

Como identificador del nuevo arquetipo openEHR se usa un nombre de espacio fhir y el nombre del recurso FHIR que el esquema openEHR modela. Por ejemplo, el identificador del arquetipo openEHR del recurso SimplePatient es:

\begin{lstlisting}
fhir::openEHR-EHR-CLUSTER.SimplePatient.v1.0.0
\end{lstlisting}

Para la sección de definición del nuevo arquetipo openEHR, se utiliza clases de openEHR para representar las estructuras de datos definidas por los tipos del esquema openEHR. La clase openEHR ELEMENT se usa para los tipos que en sus definiciones emplean clases openEHR que heredan de DATA\_VALUE, y la clase openEHR CLUSTER se usa para los demás tipos. Por ejemplo, la sección de definición del arquetipo openEHR del recurso SimplePatient usa las clases openEHR CLUSTER y openEHR ELEMENT para los tipos \(T_{SimplePatient}\) y \(T_{SimplePatient.gender}\) respectivamente:

\begin{lstlisting}[mathescape=true]
CLUSTER[id1] $\in$ {
  items $\in$ {
    ELEMENT[id2] occurrences $\in$ {0..1} $\in$ {
      value $\in$ {
        DV_TEXT[id3]
      }
    }
  }
}
\end{lstlisting}

Las definiciones de vinculación se agrega en la sección de vinculación de términos del nuevo arquetipo openEHR. Por ejemplo, la definición de vinculación del tipo \(T_{SimplePatient.gender.DV\_TEXT.value}\) del recurso SimplePatient se transcribe en la sección de vinculación de términos de su arquetipo openEHR equivalente:

\begin{lstlisting}
term_bindings = <
  [``fhir"] = <
    [``id3"] = <
http://hl7.org/ValueSet/administrative-gender
    >
  >
>
\end{lstlisting}

Un arquetipo openEHR permite la formación de caminos ADL, los cuales sirven para identificar elementos de los arquetipos openEHR \cite{openEHRArchitecture}. En esta etapa, se crea los caminos ADL del nuevo arquetipo openEHR y se relaciona con los caminos de los elementos de su recurso FHIR equivalente. Considerando el recuso de SimplePatient, la relación que se establece es:

\begin{lstlisting}[mathescape=true]
SimplePatient.gender $\Leftrightarrow$ /items[id2]/value[id3]
\end{lstlisting}


