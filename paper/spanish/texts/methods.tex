\section{Método}

La especificación de FHIR define diferentes categorías de tipos de datos. Entre los cuales, se encuentra los tipos de datos primitivos que permiten un solo valor de un dominio de valores definidos por cada tipo \cite{FHIRDataTypes}. Las demás categorías son tipos de datos complejos que utilizan a los tipos de datos primitivos para su definición.

Dentro de las especificaciones de openEHR se definen sus tipos de datos como clases de su Modelo de Referencia. Entre estas clases, se encuentra la clase DATA\_VALUE que sirve como antecesora común de todos los tipos de valores de datos en los modelos openEHR, y las clases que heredan de la misma se caracterizan por tener un atributo que almacena los tipos de datos primitivos estándares \cite{openEHRDataTypes}.

Se propone encontrar equivalencias entre los tipos de datos primitivos de FHIR y las clases que extiende de DATA\_VALUE de openEHR. Una equivalencia existe si se cumplen las condiciones de: 1) el dominio de valores de un tipo de dato primitivo de FHIR puede ser almacenado dentro de uno de los atributos de una de las clases que extiende de la clase DATA\_VALUE de openEHR, 2) se conserva el uso de la clase de openEHR para el cual fue diseñado.

Como ejemplo, si se considera el tipo primitivo de FHIR boolean y la clase DV\_BOOLEAN de openEHR se observa que: 1) los valores true, false del tipo boolean pueden ser contenidos dentro del atributo value de la clase DV\_BOOLEAN de openEHR, 2) el uso de la clase DV\_BOOLEAN, la cual especifica que la clase se usa para elementos que son datos verdaderamente booleanos se conserva. Se puede verificar, que las condiciones de equivalencia se cumplen, y, por lo tanto, según lo propuesto por este trabajo el tipo primitivo boolean de FHIR y la clase DV\_BOOLEAN de openEHR son equivalentes.

Las equivalencias encontradas se usarán para la creación de arquetipos que modelen los datos en los repositorios openEHR de recursos FHIR intercambiados.

Una forma de crear los arquetipos de recursos FHIR es modelar la misma estructura de los recursos usando las equivalencias de tipos dentro de algún editor de arquetipos disponibles \cite{openEHRModellingTools}. Los principales inconvenientes de la creación manual son la lentitud y un mayor riesgo de introducir errores por factor humano. Una alternativa diferente es la propuesta en este trabajo, consistente en la creación a través de un proceso automatizado de 3 etapas: abstracción, sustitución y definición, las cuales se realizan por cada arquetipo openEHR de recurso FHIR a crearse. La figura \ref{fig:solution} ilustra el proceso automatizado propuesto.

\begin{figure}
  \includegraphics[scale=0.5]{./images/solution}
  \caption{Proceso automatizado}
  \label{fig:solution}
\end{figure}

\subsubsection{Etapa de abstracción}

El recurso FHIR se modela dentro de un sistema de tipos similar al desarrollado en \cite{Maldonado09}. Cada elemento del recurso FHIR se abstrae por el tipo que describe su estructura. La definición de un tipo es de la forma:

\begin{align*}
T_t:=p_t\{h_t\}
\end{align*}

\noindent
donde \(T_t\) es el nombre del tipo \(t\), \(p_t\) es el predicado que describe los valores soportados por el tipo \(t\) y \(h_t\) es un CML que especifica los elementos hijos que puede tener el tipo \(t\).

La definición del recurso FHIR \(R\) con elementos \(E_1\), \dots , \(E_2\) se abstrae con la definición de tipo como sigue:

\begin{align*}
T_R:=es\_R\{T_{E_1}^{(min_{E_1} \colon max_{E_1})} \dots T_{E_2}^{(min_{E_2} \colon max_{E_2})}\}
\end{align*}

\noindent
donde \(T_{E_1}\) es el tipo que define el elemento \(E_1\), \(min_{E_1}\) y \(max_{E_1}\) son los límites inferior y superior respectivamente de las veces que se permite que el elemento \(E_1\) aparezca en el recurso \(R\). La diferencia con el CML introducido en \cite{Maldonado09} es que en la definición de un tipo no se utilizan las restricciones de longitud por no agregar información adicional a la abstracción de un recurso FHIR.

Para modelar vinculaciones, se extiende el sistema de tipos presentado en \cite{Maldonado09}, agregando definiciones de vinculación. Una definición de vinculación es de la forma:

\begin{align*}
[T_E] := CV
\end{align*}

\noindent
donde \(T_E\) es la definición del tipo del elemento al cual se le vincula el conjunto de valores \(CV\).

Por ejemplo, al considerar el recurso SimplePatient (simplificación del recurso FHIR Patient \cite{FHIRPatient}) que modela un paciente que tiene un solo elemento gender del tipo code \cite{FHIRDataTypes} vinculado al conjunto de valores AdministrativeGender \cite{FHIRAdministrativeGender}, se modela por el conjunto de tipos:

\begin{align*}
&T_{SimplePatient}:= \\
&\qquad es\_SimplePatient\{T_{SimplePatient.gender}^{(0:1)}\} \\
&T_{SimplePatient.gender}:= \\
&\qquad es\_gender\{T_{SimplePatient.gender.code}^{(1:1)}\} \\
&T_{SimplePatient.gender.code}:= \\
&\qquad es\_code\{\epsilon\} \\
&[T_{SimplePatient.gender.code}] := \\
&\qquad URL \footnotemark \\
\end{align*}

\footnotetext{https://www.hl7.org/fhir/valueset-administrative-gender.html}
Un conjunto de tipos que define un recurso FHIR se denomina esquema FHIR en este trabajo.


\subsection{Etapa de sustitución}
A partir de los esquemas FHIR se generan los esquemas openEHR. Estos son conjuntos de tipos de openEHR definidos formalmente para modelar los recursos FHIR. Dicha generación se realiza al sustituir los tipos de datos de FHIR por los de openEHR siguiendo las equivalencias encontradas. Ambos esquemas conservan la jerarquía estructural y la semántica.

Continuando con el ejemplo del recurso SimplifiedPatient al realizar el reemplazo de los tipos de FHIR por sus equivalentes de openEHR, se obtiene el esquema openEHR ilustrado en la figura \ref{fig:substitution}.

\begin{figure}
  \includegraphics[scale=0.5]{./images/substitution}
  \caption{Esquema openEHR del recurso SimplifiedPatient}
  \label{fig:substitution}
\end{figure}


\subsubsection{Etapa de definición}

Se crea un arquetipo openEHR usando Archetype Definition Language (ADL) \cite{openEHRADL} según las definiciones de un esquema openEHR.

Como identificador del nuevo arquetipo openEHR se usa un nombre de espacio fhir y el nombre del recurso FHIR que el esquema openEHR modela. Por ejemplo, el identificador del arquetipo openEHR del recurso SimplePatient es:

\begin{lstlisting}
fhir::openEHR-EHR-CLUSTER.SimplePatient.v1.0.0
\end{lstlisting}

Para la sección de definición del nuevo arquetipo openEHR, se utiliza clases de openEHR para representar las estructuras de datos definidas por los tipos del esquema openEHR. La clase openEHR ELEMENT se usa para los tipos que en sus definiciones emplean clases openEHR que heredan de DATA\_VALUE, y la clase openEHR CLUSTER se usa para los demás tipos. Por ejemplo, la sección de definición del arquetipo openEHR del recurso SimplePatient usa las clases openEHR CLUSTER y openEHR ELEMENT para los tipos \(T_{SimplePatient}\) y \(T_{SimplePatient.gender}\) respectivamente:

\begin{lstlisting}[mathescape=true]
CLUSTER[id1] $\in$ {
  items $\in$ {
    ELEMENT[id2] occurrences $\in$ {0..1} $\in$ {
      value $\in$ {
        DV_TEXT[id3]
      }
    }
  }
}
\end{lstlisting}

Las definiciones de vinculación se agrega en la sección de vinculación de términos del nuevo arquetipo openEHR. Por ejemplo, la definición de vinculación del tipo \(T_{SimplePatient.gender.DV\_TEXT.value}\) del recurso SimplePatient se transcribe en la sección de vinculación de términos de su arquetipo openEHR equivalente:

\begin{lstlisting}
term_bindings = <
  [``fhir"] = <
    [``id3"] = <
http://hl7.org/ValueSet/administrative-gender
    >
  >
>
\end{lstlisting}

Un arquetipo openEHR permite la formación de caminos ADL, los cuales sirven para identificar elementos de los arquetipos openEHR \cite{openEHRArchitecture}. En esta etapa, se crea los caminos ADL del nuevo arquetipo openEHR y se relaciona con los caminos de los elementos de su recurso FHIR equivalente. Considerando el recuso de SimplePatient, la relación que se establece es:

\begin{lstlisting}[mathescape=true]
SimplePatient.gender $\Leftrightarrow$ /items[id2]/value[id3]
\end{lstlisting}

