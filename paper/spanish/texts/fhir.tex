\subsection{FHIR}

FHIR define recursos para representar conceptos administrativos tales como paciente, proveedor, organización y dispositivo, así como conceptos clínicos que cubren problemas, medicamentos, diagnósticos, planes de atención y más \cite{FHIRResourceList}.

Los recursos tienen en común las siguientes características \cite{FHIRDeveloper}:
\begin{itemize}
  \item una URL que identifica el recurso;
  \item una metainformación común;
  \item un texto legible por el ser humano para la seguridad clínica;
  \item un conjunto definido de elementos de datos diferente para cada recurso;
  \item un marco para extender y adaptar los recursos existentes.
\end{itemize}

Los recursos se describen usando recursos StructureDefinition \cite{FHIRStructureDefinition}. Estos recursos StructureDefinition definen un conjunto de elementos de datos. Cada elemento incluye una ruta, una cardinalidad y un tipo de dato \cite{FHIRElementDefinition}.

La ruta del elemento de dato es la propiedad más importante de la definición del elemento. Esta ruta localiza el elemento en una jerarquía definida dentro del recurso \cite{FHIRElementDefinition}.

El tipo de dato del elemento de dato puede ser un tipo de dato primitivo o un tipo de dato complejo \cite{FHIRDataTypes}. La diferencia entre ambos tipos de datos es que el primitivo permite un solo valor para el elemento,  mientras que el complejo puede tener elementos hijos. Dentro de los tipos de datos complejos se encuentran: los de propósito general,  los tipos para metadatos y los de propósito especial.

Cada elemento de dato puede ser extendido o restringido \cite{FHIRProfiling}. El elemento de dato se extiende para representar información adicional que no forma parte de la definición del recurso\cite{FHIRExtensibility}. Cada recurso solo incluye elementos de datos si existe la expectativa de que la mayoría de las implementaciones usarán esos elementos de datos en particular \cite{FHIRArchitecture}. El elemento de dato se restringue cambiando su cardinalidad.

Los recursos soportan la vinculación a terminología clínica, lo cual contribuye al uso de FHIR para lograr la interoperabilidad semántica \cite{FHIRArchitecture}.

Aparte de las definiciones de los recursos, FHIR define un conjunto de interfaces por las cuales los sistemas comparten los recursos. Los mecanismos de intercambio soportados son \cite{FHIRClinician}:
\begin{itemize}
  \item a través de una interfaz REST;
  \item mediante el intercambio de documentos;
  \item vía el envío y recepción de mensajes;
  \item la exposición e invocación de servicios.
\end{itemize}
