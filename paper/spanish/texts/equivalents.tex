\subsection{Equivalencias entre tipos de datos}

La definición de un elemento en un recurso FHIR incluye tipo de dato \cite{FHIRElement}. Por lo tanto, un prerequisito es encontrar equivalencias entre tipos de datos FHIR y tipos de datos openEHR. Las equivalencias permitirán crear arquetipos openEHR a partir de definiciones de recursos FHIR.

Dada la función \( f(x) \) que retorna el dominio de valores que se le puede asignar a un dato \( x \), \( A \) el conjunto de atributos del tipo de dato de openEHR \( o \), la equivalencia entre el tipo de dato de FHIR \( p \)  y el tipo de dato de openEHR \( o \) existe si se cumplen las condiciones de:

\begin{enumerate}
  \item \( \exists a \in A \land f(a) \supseteq f(p) \);
  \item los propósitos de \( o \) y \( p \) son similares.
\end{enumerate}

Las equivalencias se encuentran al realizar una revisión manual exhaustiva del conjunto de tipos de datos de FHIR \cite{FHIRDataTypes} y del Modelo de Información de Tipos de Datos de openEHR \cite{openEHRDataTypes}. En primer lugar, para cada tipo de dato de FHIR se agrupa los tipos de datos de openEHR que tengan algún atributo cuyo dominio de valores sea un superconjunto del dominio de valores del tipo de dato de FHIR. Posteriormente, se analiza y compara las definiciones de cada tipo de dato de FHIR y su grupo de tipos de datos de openEHR encontrados inicialmente. Solo para el tipo de FHIR boolean se utiliza su definición encontrada en \cite{W3C} por no tener una definición explícita en la especificación de FHIR y por ser un tipo de dato importado desde W3C.

Como ejemplo, el tipo de FHIR boolean y el tipo de openEHR DV\_BOOLEAN reúnen ambas condiciones, por lo tanto, son equivalentes:
\begin{enumerate}
  \item los valores true y false del dominio de valores del tipo boolean pueden ser almacenados dentro del atributo value del tipo DV\_BOOLEAN;
  \item el uso de DV\_BOOLEAN, el cual especifica que el tipo se usa para elementos que son datos verdaderamente booleanos, es el mismo uso que tiene el tipo de dato boolean.
\end{enumerate}

Solo los tipos de datos primitivos de FHIR requieren una equivalencia con los tipos de datos openEHR.
