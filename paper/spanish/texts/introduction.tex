\section{Introducción}

La coordinación asistencial involucra compartir información entre todos los actores interesados en el cuidado del paciente para lograr una mejor atención \cite{CareCoordination}. Sin embargo, esta información se encuentra dispersa entre diferentes sitios, dificultando la continuidad asistencial \cite{Indarte11}. Un factor crucial para una adecuada continuidad de cuidados es la interoperabilidad de los sistemas de información que dan soporte al proceso asistencial por medio de estándares \cite{OPS16}.

En el ámbito de la salud uno de los estándares más prometedores es Fast Healthcare Interoperability Resources (FHIR) desarrollado por HL7 que combina las funcionalidades de HL7 v2, v3 y CDA con los estándares WEB (XML, JSON, HTTP, etc.) \cite{FHIR}. FHIR se basa en Recursos que son los componentes básicos para todos los intercambios \cite{FHIROverview}.

Otro estándar de amplio uso es openEHR que especifica una plataforma para construir sistemas de Historia Clínica Electrónica (EHR por sus siglas en inglés) \cite{openEHR}. En openEHR, los arquetipos son los modelos para capturar la información y uno de sus propósitos es el de garantizar la interoperabilidad \cite{Bale00}. Estos arquetipos son expresados formalmente en el Lenguaje de Definición de Arquetipos (ADL por sus siglas en inglés) \cite{ADL}. Otra parte central de openEHR son las clases de los modelos de información, conocido como Modelo de Referencia, las cuales constituyen la base del modelo de información y definen los tipos de datos y estructuras de datos soportados \cite{RM}.

En escenarios en los cuales se requiere un flujo continuo de información para el cuidado de la salud y donde existen sistemas intervinientes que implementan FHIR y openEHR \cite{Lopez16}, se necesita una estructura común en donde almacenar la información intercambiada y que permita ser entendida correctamente, procesada y utilizada de forma efectiva.

Dado que FHIR y openEHR utilizan estructuras jerárquicas basadas en tipos de datos con conjuntos de valores similares para modelar la información es posible crear una estructura común que permita la comunicación entre estos sistemas encontrando una equivalencia entre sus tipos datos. De hecho, existe colaboración entre FHIR y openEHR \cite{Collaboration} para tener un modelado de información común que por lo menos contenga todas las partes de los recursos FHIR.
