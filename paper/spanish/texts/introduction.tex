\section{Introducción}

La coordinación asistencial involucra compartir información entre todos los actores interesados en el cuidado del paciente para lograr una mejor atención \cite{CareCoordination}. Sin embargo, esta información se encuentra dispersa entre diferentes sitios, dificultando la continuidad asistencial \cite{Indarte11}.

Un factor crucial para una adecuada continuidad de cuidados es la interoperabilidad de los sistemas de información que dan soporte al proceso asistencial por medio de estándares \cite{OPS16}. Para garantizar que la información intercambiada entre sistemas de información de salud pueda ser entendida correctamente, procesada y utilizada de forma efectiva se debe estandarizar conceptos clínicos y otros conceptos de dominio usando terminologías (LOINC, SNOMED-CT, UCM, etc.) \cite{ISO20514}.

En el ámbito de la salud, uno de los estándares más prometedores es Fast Healthcare Interoperability Resources (FHIR) desarrollado por Health Level 7 (HL7). FHIR combina las funcionalidades de HL7 v2, v3 y CDA con los estándares web (XML, JSON, HTTP, OAuth, etc.) \cite{FHIR}. FHIR se basa en recursos que son los componentes básicos para todos los intercambios. Los recursos describen información clínica y administrativa. Además de definir los recursos, FHIR especifica un conjunto de interfaces, las cuales son usadas por los sistemas para compartir información. Otro estándar de amplio uso es openEHR publicado por openEHR Foundation. OpenEHR especifica una plataforma para construir sistemas EHR \cite{openEHR}. OpenEHR se basa en el modelado de dos niveles que distingue un modelo de referencia y arquetipos \cite{Bale00}. El modelo de referencia es un modelo de información estable que define las estructuras lógicas de los EHR. Los arquetipos son las definiciones formales de conceptos clínicos construidos sobre el modelo de referencia. Uno de los propósitos de los arquetipos es el de garantizar la interoperabilidad.

Algunos sistemas de salud buscan compartir información a través de repositorios de datos openEHR por medio de interfaces FHIR \cite{Lopez16}. Para ello, se necesita encontrar equivalencias entre los datos que se comparten - recursos FHIR - y el modelo de dominio de los repositorios de datos - arquetipos openEHR-. Dichas equivalencias posibilitarán, a partir de instancias de recursos FHIR, generar datos en repositorios openEHR y viceversa.

El objetivo de este trabajo es describir un proceso automatizado para crear arquetipos de integración openEHR a partir de definiciones de recursos FHIR utilizando equivalencias entre los tipos de datos de openEHR y FHIR. Estos arquetipos de integración ayudarán a intercambiar los datos entre instancias de arquetipos openEHR e instancias de recursos FHIR. El resto de este trabajo está organizado en 4 secciones. Sección 2 presenta las principales características de FHIR y openEHR, y la arquitectura de conversión de datos modelo. Sección 3 describe la creación de arquetipos de integración openEHR de recursos FHIR utilizando la metodología propuesta. Sección 4 muestra las equivalencias entre los tipos datos FHIR y openEHR, un arquetipo de integración openEHR creado con la metodología propuesta, y la comparación entre la creación automática y la manual. Finalmente, Sección 5 presenta las conclusiones del trabajo.
