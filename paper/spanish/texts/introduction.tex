\section{Introducción}

La coordinación asistencial involucra compartir información entre todos los actores interesados en el cuidado del paciente para lograr una mejor atención \cite{CareCoordination}. Sin embargo, esta información se encuentra dispersa entre diferentes sitios, dificultando la continuidad asistencial \cite{Indarte11}. Un factor crucial para una adecuada continuidad de cuidados es la interoperabilidad de los sistemas de información que dan soporte al proceso asistencial por medio de estándares \cite{OPS16}.

Para garantizar que la información intercambiada entre sistemas EHR pueda ser entendida correctamente, procesada y utilizada de forma efectiva se debe estandarizar conceptos clínicos y otros conceptos de dominio usando terminologías (LOINC, SNOMED-CT, UCM, etc) \cite{ISO20514}.

En el ámbito de la salud uno de los estándares más prometedores es Fast Healthcare Interoperability Resources (FHIR) desarrollado por HL7 que combina las funcionalidades de HL7 v2, v3 y CDA con los estándares web (XML, JSON, HTTP, OAuth, etc.) \cite{FHIR}. FHIR se basa en recursos que son los componentes básicos para todos los intercambios \cite{FHIROverview}. Además de definir los recursos, FHIR también define un conjunto de interfaces mediante el cual los sistemas comparten la información: interfaz REST, intercambio de Documentos, envío y recepción de Mensajes y la exposición e invocación de Servicios.

Otro estándar de amplio uso es openEHR que especifica una plataforma para construir sistemas de Historia Clínica Electrónica (EHR por sus siglas en inglés) \cite{openEHR}. En openEHR, los arquetipos son los modelos para capturar la información y uno de sus propósitos es el de garantizar la interoperabilidad \cite{Bale00}. Estos arquetipos son expresados formalmente en el Lenguaje de Definición de Arquetipos (ADL por sus siglas en inglés) \cite{ADL}. Otra parte central de openEHR son las clases de los modelos de información, conocido como Modelo de Referencia, las cuales constituyen la base del modelo de información y definen los tipos de datos y estructuras de datos soportados \cite{RM}.

En el escenario en el cual centros de salud requieren compartir información a través de repositorios de datos que obedecen el modelo de referencia de openEHR, y estos centros de salud implementan sobre estos repositorios interfaces de FHIR como mecanismos para el intercambio de información \cite{Lopez16}, se necesita encontrar una correspondencia entre los datos que se comparten (recursos FHIR) y modelo de información de los repositorios de datos (repositorios openEHR). Dicha relación posibilitará a partir de recursos FHIR generar datos en los repositorios openEHR y viceversa.

Dado que FHIR y openEHR utilizan estructuras jerárquicas basadas en tipos de datos con conjuntos de valores similares para modelar la información y, ambos estándares trabajan con terminologías de referencia, un enfoque para definir reglas de correspondencia directa entre los recursos FHIR y los datos en los repositorios openEHR es crear nuevos arquetipos que modelen los datos en los repositorios openEHR a partir de las definiciones de recursos FHIR intercambiados. Estos arquetipos tendrán las mismas estructuras lógicas y definiciones de terminología definidos en los recursos FHIR. Dicho enfoque es el expuesto en este trabajo.
