\section{Antecedentes}

Maldonado y col. proponen una formalización de la definición de la sección de arquetipos basada en tipos sobre una estructura de árbol. Esta formalización es independiente de la sintaxis especificada en ADL y se soporta en un sistema de tipos que modela las restricciones estructurales especificadas en los arquetipos \cite{Maldonado09}. Parte de este trabajo utilizará esta propuesta como base para la formalización.

Boscá y col. mencionan que existe una similitud entre FHIR y enfoques como el utilizado en openEHR. Tanto los recursos de FHIR y los arquetipos de openEHR definen patrones reutilizables para la descripción precisa de la información clínica. Siendo la principal diferencia que los arquetipos esperan representar la mayoría del contenido clínico, mientras que los recursos solo contienen la información clínica utilizada más común \cite{Bosca15}. Esta similitud da origen a la hipótesis de este trabajo.
