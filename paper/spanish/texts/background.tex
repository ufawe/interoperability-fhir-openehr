\section{Antecedentes}

FHIR (Fast Health Interoperability Resources) es un estándar para intercambiar información de atención médica electrónicamente. FHIR ofrece un fuerte enfoque en la implementabilidad en una amplia variedad de arquitecturas y escenarios. La versión actual de FHIR se publica como un estándar para uso experimental.

OpenEHR es un enfoque de plataforma para el desarrollo de soluciones de tecnología de la información para el cuidado de la salud. OpenEHR provee un conjunto de estándares para modelos de información clínica, extractos de EHR, datos demográficos, tipos de datos y varios tipos de interfaces de servicio \cite{openEHRWhitePaper}.

Los aspectos básicos de FHIR y openEHR se abordan a continuación.

\subsection{FHIR}

FHIR defines resources to represent administrative concepts such as patient, provider, organization and device, as well as clinical concepts that cover problems, drugs, diagnostics, care plans and more \cite{FHIRResourceList}.

Resources have the following characteristics in common \cite{FHIRDeveloper}:
\begin{itemize}
  \item a URL that identifies the resource;
  \item common meta-information;
  \item a human-readable text, for clinical safety;
  \item a defined set of data elements, different for each resource;
  \item a framework to extend and adapt existing resources.
\end{itemize}

Resources are described using StructureDefinition resource \cite{FHIRStructureDefinition}. These StructureDefinition resources define a set of data elements. Each element includes a path, cardinality and a type of data \cite{FHIRElementDefinition}.

Data element path is the most important property of the element's definition. This path locates the element in a hierarchy defined within the resource \cite{FHIRElementDefinition}.

The data type of the data element may be primitive or complex  \cite{FHIRDataTypes}. The difference between both types of data is that the primitive type allows a single value for the element, while the complex type may have child elements. Each type of primitive data is a 3-tuple, which consists of: a) a domain of values, which includes the definition of the data type, b) a XML representation and c) a JSON representation. Within the complex data types there are types with general purpose, meta-data types and special purpose types.

Each data element can be extended or constrained \cite{FHIRProfiling}. Data element extends to represent additional information that is not a part of the resource definition \cite{FHIRExtensibility}. Each resource will only include data elements if those particular elements will be used by the majority of the implementations \cite{FHIRArchitecture}. The data element is restricted by changing its cardinality.

The resources support binding to clinical terminology, which contributes to achieving semantic interoperability \cite{FHIRArchitecture}.

Other than resource definitions, FHIR defines a set of interfaces through which the systems share the resources. The supported exchange mechanisms are the following \cite{FHIRClinician}:
\begin{itemize}
  \item through RESTful API;
  \item by means of documents exchanges;
  \item via messaging;
  \item Services / SOA.
\end{itemize}


\subsection{OpenEHR}

El enfoque de openEHR es un riguroso modelado del conocimiento, y se basa en el principio básico de separación de los dominios de concepto y los dominios de información en los sistemas de información. Los dominios de concepto son modelados usando los arquetipos y expresados en el Lenguaje de Definición de Arquetipos (ADL).

\subsubsection{Modelado en openEHR}

El enfoque de openEHR para el modelado es multinivel. Los modelos son desarrollados y mantenidos por expertos de dominio en su propio nivel \cite{openEHRArchitecture}.

El primer nivel se basa en el modelo de referencia. El modelo de referencia corresponde al modelo de información estable, por ejemplo, tipos de datos o estructuras de datos. Todos los datos EHR en cualquier sistema openEHR obedecen el modelo de referencia. Solo este primer nivel es implementado en software \cite{openEHRArchitecture}.

El siguiente nivel consiste en los arquetipos. Los arquetipos corresponden a los contenidos del dominio, por ejemplo, medidas de la presión sanguínea o resultado de la prueba para diabetes. Estos arquetipos son modelados por profesionales clínicos o expertos en informática de la salud sin ningún conocimiento tecnológico de los sistemas finales. Los arquetipos son almacenados en sus propios repositorios.

Las plantillas constituyen el siguiente nivel. Estas plantillas especifican grupos de arquetipos que se usan para un propósito particular, y a menudo corresponden a formularios de pantalla \cite{openEHRArchitecture}.

En el último nivel se encuentran los artefactos generados a partir de las plantillas \cite{openEHR}. Los artefactos, mostrados en la Figura \ref{fig:openeEHR_ecosystem}, no se modelan manualmente. Esto significa que un modelo para resultado de microbiología solo necesita hacerse una vez para habilitar los informes, formularios de interfaz de usuario, documentos y otros formatos de mensaje que se generarán.

\begin{figure}[h]
  \centering
  \includegraphics[scale=0.6]{./images/openehr_dev_ecosystem}
  \caption{Enfoque openEHR (Fuente: Extraído desde \cite{openEHR})}
  \label{fig:openeEHR_ecosystem}
\end{figure}


\subsubsection{Arquetipos}

Cada arquetipo es un conjunto de restricciones en el modelo de referencia, definiendo un contenido de dominio \cite{openEHRArchitecture}. La definición de un arquetipo consiste en capas alternativas de nodos de restricción de objeto y atributo \cite{openEHRAOM}. Los diferentes nodos son:
\begin{itemize}
  \item nodos de restricción de primitivos que restringen tipos de datos primitivos;
  \item nodos que representan referencias a otros nodos;
  \item nodos de referencia de restricción que hacen referencia a una restricción de texto en la parte de vinculación de restricción de la terminología del arquetipo;
  \item nodos de restricción de arquetipo que representan restricciones en otros arquetipos permitidos para aparecer en un punto dado.
\end{itemize}

Las restricciones de los nodos definen que configuraciones de instancias de la clase de modelo de referencia se consideran conformes al arquetipo. Por ejemplo, ciertas configuraciones de las clases PARTY, ADDRESS, CLUSTER y ELEMENT pueden definirse por un arquetipo Person como estructuras permitidas para personas con identidad, contactos y direcciones.

Los arquetipos pueden tener relaciones de especialización y/o composición. Los arquetipos especializados son creados restringiendo aún más las restricciones existentes de otros arquetipos. Los arquetipos compuestos son definidos a partir de otros arquetipos más pequeños \cite{openEHRArchitecture}.

OpenEHR incluye un mecanismo de rutas. Las rutas son construidas usando la jerarquía de nodos objeto y atributo de un arquetipo en una sintaxis compatible con Xpath. Estas rutas pueden usarse para referenciar a cualquier nodo en un arquetipo \cite{openEHRArchitecture}.

Los arquetipos proveen una forma de definir el significado de los datos, y de conectar los datos a terminologías conocidas como SNOMED CT, LOINC, ICPC, ICDx y muchas otras terminologías y vocabularios usados en salud \cite{openEHRArchitecture}.


\subsubsection{Lenguaje de Definición de Arquetipo}

Los arquetipos son expresados en el Lenguaje de Definición de Arquetipo \cite{openEHRADL} (ADL por sus siglas en inglés). ADL utiliza tres sintaxis: cADL (forma de restricción de ADL), ODIN (notación de instancia de datos de objeto) y una versión de lógica de predicado de primer orden (FOPL por sus siglas en inglés).

Las restricciones de cADL se escriben en un estilo estructurado en bloques, similar a los lenguajes de programación estructurados en bloques como C. Cada bloque se introduce mediante un identificador de un modelo de información de openEHR. Los identificadores alternan entre los nombres de tipo conocidos como bloques de objetos o nodos de objeto y los nombres de atributos de tipo conocidos como bloques de atributo o nodos de atributos. El uso de nodos de objeto permite la formación de rutas de arquetipo, que se pueden utilizar para hacer de forma inequívoca a nodos de objetos dentro del mismo arquetipo \cite{openEHRADL}.

La sintaxis de cADL se utiliza para expresar la definición de los arquetipos, mientras que la sintaxis de ODIN se usa para expresar datos que aparecen en las secciones de idioma, descripción, terminología y revisión histórica de un arquetipo.

Actualmente hay dos versiones principales existentes: `ADL 1.4', la versión original, y `ADL 2', una versión más moderna, que se está adoptando lentamente.



FHIR y openEHR comparten cierta similitud. Los recursos de FHIR y los arquetipos de openEHR definen patrones reutilizables para la descripción precisa de la información clínica \cite{Bosca15}. Sin embargo, los trabajos de colaboración entre las comunidades de FHIR y openEHR \cite{Collaboration} no consiguieron generar recursos y arquetipos con un contenido coincidente y clínicamente verificable. Una de las causas son los principios de diseño diferentes utilizados por ambas comunidades. Siendo la principal diferencia que los arquetipos esperan representar la mayoría del contenido clínico, mientras que los recursos solo contienen la información clínica utilizada más común.

El enfoque propuesto en este trabajo utiliza los recursos FHIR StructureDefinition para crear nuevos arquetipos openEHR. Estos nuevos arquetipos, expresados en ADL, facilitarán el intercambio de datos entre sistemas FHIR y sistemas openEHR. El intercambio se logra estableciendo correspondencias entre las rutas de los elementos de los recursos y las rutas de los nodos de los arquetipos. Además, el enfoque propuesto hace uso del soporte de ambos estándares a terminologías para conservar el significado de los datos intercambiados.
