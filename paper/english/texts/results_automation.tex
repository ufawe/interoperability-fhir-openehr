\subsection{Automation}

FHIR resource element definitions are found in the snapshot attribute of a StructureDefinition FHIR resource. A FHIR resource element is defined in an ElemenDefinition FHIR resource \cite{FHIRElementDefinition}. FHIR StructureDefinition resources as well as FHIR ElementDefinition resources have representations in JSON. These representations are used in the automation process to create an openEHR archetype out of a FHIR resource in the following manner.

First, in the abstraction stage, a FHIR resource name is obtained from a FHIR StructureDefinition resource id attribute. For each element defined in a FHIR StructureDefinition resource, element name, element type, minimum cardinality and maximum cardinality are obtained from a FHIR ElementDefinition resource's  id, type, min and max attributes, respectively. With this data, a definition of an element type is created with the method described above. A binding definition is created out of a binding attribute from a FHIR ElementDefinition resource.

During the substitution stage, data types are replaced by their equivalent using equivalences listed in table \ref{table:equivalents}.

Finally, in the definition stage, ADL CLUSTER and ELEMENT blocks are created according to the explanation above, with the following considerations. Given that class openEHR DV\_QUANTITY requires the use of units expressed in UCUM \cite{UCUM}, ], default value is used in 1 within the units attribute to denote without units. Taking into account that class openEHR DV\_PARSABLE requires specification of the used formalism, values base64Binary and Markdown are used for FHIR base64Binary and Markdown types, respectively. An important aspect is that recursive definitions in FHIR are not supported in openEHR archetypes. These recursive definitions are limited to level 1 depth in the definition stage. An implementation of the automated process in python language is available in \cite{PythonImplementation}.

For the purpose of comparing manual and automatic creation time, an experiment was carried out in which time required for the manual creation fo openEHR archetype of SimplePatient resource, presented in this work was compared to the automatic creation of openEHR archetypes of FHIR resources categorized as individuals. ADLWORKbench editor was used for manual creation \cite{ADLWORKbench}.The process in figure \ref{fig:solution} implemented in python language, was used for automatic creation \cite{PythonImplementation}. Experiment results showed that automatic creation of openEHR archetypes of all resources categorized as individuals (6 resources) took 281.53 ms in average, facing the average 79.5 seconds it took for the manual creation of SimplePatient resource definition.
